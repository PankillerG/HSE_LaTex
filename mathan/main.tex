\documentclass{article}
\usepackage[T2A, T1]{fontenc}
\usepackage[utf8]{inputenc}
\usepackage[russian]{babel}
\usepackage{titling}
\usepackage{amsmath}
\usepackage{mathtools}
\usepackage{amsthm}
\usepackage{python}
% \usepackage{minted}
\usepackage{amssymb}
\usepackage{amsthm}
\usepackage{amsthm}
\usepackage{hyperref}
\usepackage{listings}
\usepackage{xcolor}

\usepackage{graphicx}
\graphicspath{ {./} }

\hypersetup{
    colorlinks,
    citecolor=black,
    filecolor=black,
    linkcolor=black,
    urlcolor=black
}
\usepackage{titlesec}
\usepackage[rightcaption]{sidecap}
\usepackage{wrapfig}
\titleformat*{\subsubsection}{\normalfont}

\makeatletter
\renewcommand*\env@matrix[1][*\c@MaxMatrixCols c]{%
  \hskip -\arraycolsep
  \let\@ifnextchar\new@ifnextchar
  \array{#1}}
\makeatother


\setlength{\droptitle}{-3.5cm}
\setlength{\parindent}{0cm}
\newcommand{\squad}{\hspace{0.5em}}
\renewcommand{\arraystretch}{1.5}
\newcommand*{\bigchi}{\mbox{\Large$\chi$}}

\newtheorem{definition}{Определение}
\newtheorem{theorem}{Теорема}
\newtheorem{lemma}{Лемма}
\newtheorem{statement}{Утверждение}
% \renewcommand{\sectionbreak}{}


\author{hse-ami}
\title{Лекции по математическому анализу. ФКН 2 курс.}
\date{}
\usepackage
[
        a4paper,
        left=2cm,
        right=2cm,
        top=3cm,
        bottom=4cm
]
{geometry}

\begin{document}

\maketitle
\tableofcontents{}
\newpage

\section{Лекция 1 (18.01)}

\large \faYoutube \normalsize $\>$ \url{https://www.youtube.com/watch?v=qr_1zepmqBY}

\begin{definition}
    \textbf{Дифференциальным} называется уравнение, которое связывает значение функции с ее производной.
\end{definition}

\begin{definition}
    \textbf{Обыкновенное дифференциальное уравнение} (ОДУ) -- это уравнение, зависящее от одной независимой переменной, т.е. $x(t)$. Данный тип уравнений содержит обыкновенные производные.
    \[
        P(t,x) dt + Q(x,t) dx = 0
    \]
\end{definition}

\begin{definition}
    \textbf{Дифференциальные уравнения в частных производных} (УРЧП) -- это уравнения,содержащие неизвестные функции от нескольких переменных и их частные производные, т.е. $v(x,y,z,t)$. Данный тип уравнений содержит частные производные.
    \[
        P(x_1, x_2, ..., x_m, z, \frac{\delta z}{\delta x_1}, ..., \frac{\delta^n z}{\delta x_m^n}) = 0
    \]
    Решение УРЧП обычно сложнее, чем решение ОДУ.
\end{definition}

\subsection{Обыкновенные дифференциальные уравнения.}

Обыкновенное дифференциальное уравнение (ОДУ) может быть интегрировано напрямую:
\[
    \frac{d^n x}{d t^n} = G(t),
\]
где производная $x = x(t)$ может быть любого порядка, а правая часть уравнения может зависеть только от независимой переменной $t$.

Пример:
\[
    m \frac{d^2 x}{d t^2} = -mg,
\]
где $x$ -- высота объекта над уровнем земли, $m$ -- его масса, $g = 9.81 m/s^2$ -- ускорение свободного падения.
Массу можно сократить:
\[
    \frac{d^2 x}{d t^2} = -g
\]

\subsection{Модель экономического роста Роберта Солоу.}
\[
    \frac{dk}{dt} = sf(k) - \delta k,
\]
где $\frac{dk}{dt}$ -- темп изменения значения функции $k(t)$ (капитала), $s$ -- ставка сбережения, $\delta$ -- темп амортизации, $f$ -- функция производства.
\newpage

\subsection{Дифференциальные уравнения первого порядка.}

Классический вид дифференциальных уравнений первого порядка для функции $y = y(t)$
\[
    \frac{dy}{dx} = f(x,y),
\]
где $f(x,y)$ -- функция от независимой переменной $x$ и зависимой переменной $y$.
\newline
Нахождение числового решения уравнения:
\newline
Рассмотрим дифференциальное уравнения первого порядка следующего вида
\[
    a_0(x) y^{(n)} (x) + a_1(x) y^{(n-1)} (x) + ... + a_n(x_n) = F(x)
\]
в случае решения системы ДУ или УРЧП, каждое уравнение должно иметь линейную форму. Когда ДУ невозможно привести к линейному виду, такое уравнение называется нелинейным ДУ.
\newline
Если функция $F$ больше нуля (положительна), то такое уравнения называется однородным, в противном случае -- неоднородным.
\newline
Если ДУ не содержит независимых переменных, оно называется автономным ДУ.

\subsection{Метод Эйлера.}
Не всегда возможно для уравнения $y = y(t)$ найти решение аналитически, но всегда можно найти численное решение уравнения, зная, что функция $f(x,y)$ -- это функция с регулярным поведением, а так же некое его значение $y(x_0) = y_0$. ДУ показывает касательную к функции в какой-то конкретной точке $(x_0, y_0)$, где производная функции означает ее наклон.
\newline
Используя метод Эйлера, получим $$y_1 = y_0 + \Delta x f(x_0, y_0)$$
Решение уравнения $(x_1, y_1)$ станет начальным условием, справедливым для следующей точки $(x_2, y_2) $ определяющимся новым наклоном касательной $f(x_1, y_1)$. Для малых значений $\Delta x$ численное решение сходится к точному решению уравнения.

\subsection{Дифференциальные уравнения с разделяющимися переменными.}
Дифференциальные уравнения с разделяющимися переменными имеют следующий вид:
\[
    g(y) \frac{dy}{dx} = f(x), \quad y(x_0) = y_0,
\]
где функция $g(y)$ не зависит от переменной $x$, а $f(x)$ от $y$. Интегрируем от $x_0$ до $x$:
\[
    \int \limits_{x_0}^x g(y(x))y'(x) dx = \int \limits_{x_0}^x f(x) dx
\]
Используем замену $u = y(x), dy = y'(x)dx$, меняем пределы интегрирования относительно $y$:
\[
    \int \limits_{y_0}^{y} g(u) du = \int \limits_{x_0}^x f(x) dx
\]
\newpage
\section{Лекция 2 (дата)}
\newpage
\section{Лекция 3 (дата)}
\newpage
\section{Лекция 4 (дата)}
\newpage


\end{document}