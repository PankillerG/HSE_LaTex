\section{Лекция 1 (18.01)}

\large \faYoutube \normalsize $\>$ \url{https://www.youtube.com/watch?v=qr_1zepmqBY}

\begin{definition}
    \textbf{Дифференциальным} называется уравнение, которое связывает значение функции с ее производной.
\end{definition}

\begin{definition}
    \textbf{Обыкновенное дифференциальное уравнение} (ОДУ) -- это уравнение, зависящее от одной независимой переменной, т.е. $x(t)$. Данный тип уравнений содержит обыкновенные производные.
    \[
        P(t,x) dt + Q(x,t) dx = 0
    \]
\end{definition}

\begin{definition}
    \textbf{Дифференциальные уравнения в частных производных} (УРЧП) -- это уравнения,содержащие неизвестные функции от нескольких переменных и их частные производные, т.е. $v(x,y,z,t)$. Данный тип уравнений содержит частные производные.
    \[
        P(x_1, x_2, ..., x_m, z, \frac{\delta z}{\delta x_1}, ..., \frac{\delta^n z}{\delta x_m^n}) = 0
    \]
    Решение УРЧП обычно сложнее, чем решение ОДУ.
\end{definition}

\subsection{Обыкновенные дифференциальные уравнения.}

Обыкновенное дифференциальное уравнение (ОДУ) может быть интегрировано напрямую:
\[
    \frac{d^n x}{d t^n} = G(t),
\]
где производная $x = x(t)$ модет быть любого порядка, а правая часть уравнения может зависеть только от независимой переменной $t$.